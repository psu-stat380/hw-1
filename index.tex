% Options for packages loaded elsewhere
\PassOptionsToPackage{unicode}{hyperref}
\PassOptionsToPackage{hyphens}{url}
\PassOptionsToPackage{dvipsnames,svgnames,x11names}{xcolor}
%
\documentclass[
  letterpaper,
  DIV=11,
  numbers=noendperiod]{scrartcl}

\usepackage{amsmath,amssymb}
\usepackage{lmodern}
\usepackage{iftex}
\ifPDFTeX
  \usepackage[T1]{fontenc}
  \usepackage[utf8]{inputenc}
  \usepackage{textcomp} % provide euro and other symbols
\else % if luatex or xetex
  \usepackage{unicode-math}
  \defaultfontfeatures{Scale=MatchLowercase}
  \defaultfontfeatures[\rmfamily]{Ligatures=TeX,Scale=1}
\fi
% Use upquote if available, for straight quotes in verbatim environments
\IfFileExists{upquote.sty}{\usepackage{upquote}}{}
\IfFileExists{microtype.sty}{% use microtype if available
  \usepackage[]{microtype}
  \UseMicrotypeSet[protrusion]{basicmath} % disable protrusion for tt fonts
}{}
\makeatletter
\@ifundefined{KOMAClassName}{% if non-KOMA class
  \IfFileExists{parskip.sty}{%
    \usepackage{parskip}
  }{% else
    \setlength{\parindent}{0pt}
    \setlength{\parskip}{6pt plus 2pt minus 1pt}}
}{% if KOMA class
  \KOMAoptions{parskip=half}}
\makeatother
\usepackage{fancyvrb}
\usepackage{xcolor}
\setlength{\emergencystretch}{3em} % prevent overfull lines
\setcounter{secnumdepth}{-\maxdimen} % remove section numbering
% Make \paragraph and \subparagraph free-standing
\ifx\paragraph\undefined\else
  \let\oldparagraph\paragraph
  \renewcommand{\paragraph}[1]{\oldparagraph{#1}\mbox{}}
\fi
\ifx\subparagraph\undefined\else
  \let\oldsubparagraph\subparagraph
  \renewcommand{\subparagraph}[1]{\oldsubparagraph{#1}\mbox{}}
\fi

\usepackage{color}
\usepackage{fancyvrb}
\newcommand{\VerbBar}{|}
\newcommand{\VERB}{\Verb[commandchars=\\\{\}]}
\DefineVerbatimEnvironment{Highlighting}{Verbatim}{commandchars=\\\{\}}
% Add ',fontsize=\small' for more characters per line
\usepackage{framed}
\definecolor{shadecolor}{RGB}{241,243,245}
\newenvironment{Shaded}{\begin{snugshade}}{\end{snugshade}}
\newcommand{\AlertTok}[1]{\textcolor[rgb]{0.68,0.00,0.00}{#1}}
\newcommand{\AnnotationTok}[1]{\textcolor[rgb]{0.37,0.37,0.37}{#1}}
\newcommand{\AttributeTok}[1]{\textcolor[rgb]{0.40,0.45,0.13}{#1}}
\newcommand{\BaseNTok}[1]{\textcolor[rgb]{0.68,0.00,0.00}{#1}}
\newcommand{\BuiltInTok}[1]{\textcolor[rgb]{0.00,0.23,0.31}{#1}}
\newcommand{\CharTok}[1]{\textcolor[rgb]{0.13,0.47,0.30}{#1}}
\newcommand{\CommentTok}[1]{\textcolor[rgb]{0.37,0.37,0.37}{#1}}
\newcommand{\CommentVarTok}[1]{\textcolor[rgb]{0.37,0.37,0.37}{\textit{#1}}}
\newcommand{\ConstantTok}[1]{\textcolor[rgb]{0.56,0.35,0.01}{#1}}
\newcommand{\ControlFlowTok}[1]{\textcolor[rgb]{0.00,0.23,0.31}{#1}}
\newcommand{\DataTypeTok}[1]{\textcolor[rgb]{0.68,0.00,0.00}{#1}}
\newcommand{\DecValTok}[1]{\textcolor[rgb]{0.68,0.00,0.00}{#1}}
\newcommand{\DocumentationTok}[1]{\textcolor[rgb]{0.37,0.37,0.37}{\textit{#1}}}
\newcommand{\ErrorTok}[1]{\textcolor[rgb]{0.68,0.00,0.00}{#1}}
\newcommand{\ExtensionTok}[1]{\textcolor[rgb]{0.00,0.23,0.31}{#1}}
\newcommand{\FloatTok}[1]{\textcolor[rgb]{0.68,0.00,0.00}{#1}}
\newcommand{\FunctionTok}[1]{\textcolor[rgb]{0.28,0.35,0.67}{#1}}
\newcommand{\ImportTok}[1]{\textcolor[rgb]{0.00,0.46,0.62}{#1}}
\newcommand{\InformationTok}[1]{\textcolor[rgb]{0.37,0.37,0.37}{#1}}
\newcommand{\KeywordTok}[1]{\textcolor[rgb]{0.00,0.23,0.31}{#1}}
\newcommand{\NormalTok}[1]{\textcolor[rgb]{0.00,0.23,0.31}{#1}}
\newcommand{\OperatorTok}[1]{\textcolor[rgb]{0.37,0.37,0.37}{#1}}
\newcommand{\OtherTok}[1]{\textcolor[rgb]{0.00,0.23,0.31}{#1}}
\newcommand{\PreprocessorTok}[1]{\textcolor[rgb]{0.68,0.00,0.00}{#1}}
\newcommand{\RegionMarkerTok}[1]{\textcolor[rgb]{0.00,0.23,0.31}{#1}}
\newcommand{\SpecialCharTok}[1]{\textcolor[rgb]{0.37,0.37,0.37}{#1}}
\newcommand{\SpecialStringTok}[1]{\textcolor[rgb]{0.13,0.47,0.30}{#1}}
\newcommand{\StringTok}[1]{\textcolor[rgb]{0.13,0.47,0.30}{#1}}
\newcommand{\VariableTok}[1]{\textcolor[rgb]{0.07,0.07,0.07}{#1}}
\newcommand{\VerbatimStringTok}[1]{\textcolor[rgb]{0.13,0.47,0.30}{#1}}
\newcommand{\WarningTok}[1]{\textcolor[rgb]{0.37,0.37,0.37}{\textit{#1}}}

\providecommand{\tightlist}{%
  \setlength{\itemsep}{0pt}\setlength{\parskip}{0pt}}\usepackage{longtable,booktabs,array}
\usepackage{calc} % for calculating minipage widths
% Correct order of tables after \paragraph or \subparagraph
\usepackage{etoolbox}
\makeatletter
\patchcmd\longtable{\par}{\if@noskipsec\mbox{}\fi\par}{}{}
\makeatother
% Allow footnotes in longtable head/foot
\IfFileExists{footnotehyper.sty}{\usepackage{footnotehyper}}{\usepackage{footnote}}
\makesavenoteenv{longtable}
\usepackage{graphicx}
\makeatletter
\def\maxwidth{\ifdim\Gin@nat@width>\linewidth\linewidth\else\Gin@nat@width\fi}
\def\maxheight{\ifdim\Gin@nat@height>\textheight\textheight\else\Gin@nat@height\fi}
\makeatother
% Scale images if necessary, so that they will not overflow the page
% margins by default, and it is still possible to overwrite the defaults
% using explicit options in \includegraphics[width, height, ...]{}
\setkeys{Gin}{width=\maxwidth,height=\maxheight,keepaspectratio}
% Set default figure placement to htbp
\makeatletter
\def\fps@figure{htbp}
\makeatother

\KOMAoption{captions}{tableheading}
\makeatletter
\@ifpackageloaded{tcolorbox}{}{\usepackage[many]{tcolorbox}}
\@ifpackageloaded{fontawesome5}{}{\usepackage{fontawesome5}}
\definecolor{quarto-callout-color}{HTML}{909090}
\definecolor{quarto-callout-note-color}{HTML}{0758E5}
\definecolor{quarto-callout-important-color}{HTML}{CC1914}
\definecolor{quarto-callout-warning-color}{HTML}{EB9113}
\definecolor{quarto-callout-tip-color}{HTML}{00A047}
\definecolor{quarto-callout-caution-color}{HTML}{FC5300}
\definecolor{quarto-callout-color-frame}{HTML}{acacac}
\definecolor{quarto-callout-note-color-frame}{HTML}{4582ec}
\definecolor{quarto-callout-important-color-frame}{HTML}{d9534f}
\definecolor{quarto-callout-warning-color-frame}{HTML}{f0ad4e}
\definecolor{quarto-callout-tip-color-frame}{HTML}{02b875}
\definecolor{quarto-callout-caution-color-frame}{HTML}{fd7e14}
\makeatother
\makeatletter
\makeatother
\makeatletter
\makeatother
\makeatletter
\@ifpackageloaded{caption}{}{\usepackage{caption}}
\AtBeginDocument{%
\ifdefined\contentsname
  \renewcommand*\contentsname{Table of contents}
\else
  \newcommand\contentsname{Table of contents}
\fi
\ifdefined\listfigurename
  \renewcommand*\listfigurename{List of Figures}
\else
  \newcommand\listfigurename{List of Figures}
\fi
\ifdefined\listtablename
  \renewcommand*\listtablename{List of Tables}
\else
  \newcommand\listtablename{List of Tables}
\fi
\ifdefined\figurename
  \renewcommand*\figurename{Figure}
\else
  \newcommand\figurename{Figure}
\fi
\ifdefined\tablename
  \renewcommand*\tablename{Table}
\else
  \newcommand\tablename{Table}
\fi
}
\@ifpackageloaded{float}{}{\usepackage{float}}
\floatstyle{ruled}
\@ifundefined{c@chapter}{\newfloat{codelisting}{h}{lop}}{\newfloat{codelisting}{h}{lop}[chapter]}
\floatname{codelisting}{Listing}
\newcommand*\listoflistings{\listof{codelisting}{List of Listings}}
\makeatother
\makeatletter
\@ifpackageloaded{caption}{}{\usepackage{caption}}
\@ifpackageloaded{subcaption}{}{\usepackage{subcaption}}
\makeatother
\makeatletter
\@ifpackageloaded{tcolorbox}{}{\usepackage[many]{tcolorbox}}
\makeatother
\makeatletter
\@ifundefined{shadecolor}{\definecolor{shadecolor}{rgb}{.97, .97, .97}}
\makeatother
\makeatletter
\makeatother
\ifLuaTeX
  \usepackage{selnolig}  % disable illegal ligatures
\fi
\IfFileExists{bookmark.sty}{\usepackage{bookmark}}{\usepackage{hyperref}}
\IfFileExists{xurl.sty}{\usepackage{xurl}}{} % add URL line breaks if available
\urlstyle{same} % disable monospaced font for URLs
\VerbatimFootnotes % allow verbatim text in footnotes
\hypersetup{
  pdftitle={Homework 1},
  pdfauthor={Miranda Goodman},
  colorlinks=true,
  linkcolor={blue},
  filecolor={Maroon},
  citecolor={Blue},
  urlcolor={Blue},
  pdfcreator={LaTeX via pandoc}}

\title{Homework 1}
\author{{Miranda Goodman}}
\date{}

\begin{document}
\maketitle
\ifdefined\Shaded\renewenvironment{Shaded}{\begin{tcolorbox}[frame hidden, borderline west={3pt}{0pt}{shadecolor}, boxrule=0pt, sharp corners, breakable, enhanced, interior hidden]}{\end{tcolorbox}}\fi

\renewcommand*\contentsname{Table of contents}
{
\hypersetup{linkcolor=}
\setcounter{tocdepth}{3}
\tableofcontents
}
\href{https://github.com/psu-stat380/hw-1}{Link to the Github
repository}

\begin{center}\rule{0.5\linewidth}{0.5pt}\end{center}

\begin{tcolorbox}[enhanced jigsaw, opacityback=0, rightrule=.15mm, leftrule=.75mm, colframe=quarto-callout-important-color-frame, bottomrule=.15mm, opacitybacktitle=0.6, colbacktitle=quarto-callout-important-color!10!white, toptitle=1mm, coltitle=black, bottomtitle=1mm, left=2mm, breakable, arc=.35mm, titlerule=0mm, title=\textcolor{quarto-callout-important-color}{\faExclamation}\hspace{0.5em}{Due: Sun, Jan 29, 2023 @ 11:59pm}, toprule=.15mm, colback=white]

Please read the instructions carefully before submitting your
assignment.

\begin{enumerate}
\def\labelenumi{\arabic{enumi}.}
\item
  This assignment requires you to:

  \begin{itemize}
  \tightlist
  \item
    Upload your Quarto markdown files to a \texttt{git} repository
  \item
    Upload a \texttt{PDF} file on Canvas
  \end{itemize}
\item
  Don't collapse any code cells before submitting.
\item
  Remember to make sure all your code output is rendered properly before
  uploading your submission.
\end{enumerate}

⚠️ Please add your name to the the author information in the frontmatter
before submitting your assignment.

\end{tcolorbox}

\begin{center}\rule{0.5\linewidth}{0.5pt}\end{center}

\hypertarget{question-1}{%
\subsection{Question 1}\label{question-1}}

\begin{tcolorbox}[enhanced jigsaw, opacityback=0, rightrule=.15mm, leftrule=.75mm, colframe=quarto-callout-tip-color-frame, bottomrule=.15mm, opacitybacktitle=0.6, colbacktitle=quarto-callout-tip-color!10!white, toptitle=1mm, coltitle=black, bottomtitle=1mm, left=2mm, breakable, arc=.35mm, titlerule=0mm, title=\textcolor{quarto-callout-tip-color}{\faLightbulb}\hspace{0.5em}{20 points}, toprule=.15mm, colback=white]

\end{tcolorbox}

In this question, we will walk through the process of \emph{forking} a
\texttt{git} repository and submitting a \emph{pull request}.

\begin{enumerate}
\def\labelenumi{\arabic{enumi}.}
\tightlist
\item
  Navigate to the Github repository
  \href{https://github.com/psu-stat380/hw-1}{here} and fork it by
  clicking on the icon in the top right
\end{enumerate}

\includegraphics{images/fork.png}

\begin{quote}
Provide a sensible name for your forked repository when prompted.
\end{quote}

\begin{enumerate}
\def\labelenumi{\arabic{enumi}.}
\setcounter{enumi}{1}
\item
  Clone your Github repository on your local machine

\begin{Shaded}
\begin{Highlighting}[]
\ExtensionTok{$}\NormalTok{ git clone }\OperatorTok{\textless{}\textless{}https://github.com/psu{-}stat380/hw{-}1.git\textgreater{}\textgreater{}}
\StringTok{$ cd hw{-}1}
\end{Highlighting}
\end{Shaded}

  Alternatively, you can use Github codespaces to get started from your
  repository directly.
\item
  In order to activate the \texttt{R} environment for the homework, make
  sure you have \texttt{renv} installed beforehand. To activate the
  \texttt{renv} environment for this assignment, open an instance of the
  \texttt{R} console from within the directory and type

\begin{Shaded}
\begin{Highlighting}[]
\NormalTok{renv}\SpecialCharTok{::}\FunctionTok{activate}\NormalTok{()}
\end{Highlighting}
\end{Shaded}

  Follow the instrutions in order to make sure that \texttt{renv} is
  configured correctly.
\item
  Work on the \emph{reminaing part} of this assignment as a
  \texttt{.qmd} file.

  \begin{itemize}
  \tightlist
  \item
    Create a \texttt{PDF} and \texttt{HTML} file for your output by
    modifying the YAML frontmatter for the Quarto \texttt{.qmd} document
  \end{itemize}
\item
  When you're done working on your assignment, push the changes to your
  github repository.
\item
  Navigate to the original Github repository
  \href{https://github.com/psu-stat380/hw-1}{here} and submit a pull
  request linking to your repository.

  Remember to \textbf{include your name} in the pull request
  information!
\end{enumerate}

If you're stuck at any step along the way, you can refer to the
\href{https://docs.github.com/en/pull-requests/collaborating-with-pull-requests/proposing-changes-to-your-work-with-pull-requests/creating-a-pull-request-from-a-fork}{official
Github docs here}

\begin{center}\rule{0.5\linewidth}{0.5pt}\end{center}

\hypertarget{question-2}{%
\subsection{Question 2}\label{question-2}}

\begin{tcolorbox}[enhanced jigsaw, opacityback=0, rightrule=.15mm, leftrule=.75mm, colframe=quarto-callout-tip-color-frame, bottomrule=.15mm, opacitybacktitle=0.6, colbacktitle=quarto-callout-tip-color!10!white, toptitle=1mm, coltitle=black, bottomtitle=1mm, left=2mm, breakable, arc=.35mm, titlerule=0mm, title=\textcolor{quarto-callout-tip-color}{\faLightbulb}\hspace{0.5em}{30 points}, toprule=.15mm, colback=white]

\end{tcolorbox}

Consider the following vector

\begin{Shaded}
\begin{Highlighting}[]
\NormalTok{my\_vec }\OtherTok{\textless{}{-}} \FunctionTok{c}\NormalTok{(}
    \StringTok{"+0.07"}\NormalTok{,}
    \StringTok{"{-}0.07"}\NormalTok{,}
    \StringTok{"+0.25"}\NormalTok{,}
    \StringTok{"{-}0.84"}\NormalTok{,}
    \StringTok{"+0.32"}\NormalTok{,}
    \StringTok{"{-}0.24"}\NormalTok{,}
    \StringTok{"{-}0.97"}\NormalTok{,}
    \StringTok{"{-}0.36"}\NormalTok{,}
    \StringTok{"+1.76"}\NormalTok{,}
    \StringTok{"{-}0.36"}
\NormalTok{)}
\end{Highlighting}
\end{Shaded}

For the following questions, provide your answers in a code cell.

\begin{enumerate}
\def\labelenumi{\arabic{enumi}.}
\tightlist
\item
  What data type does the vector contain?
\end{enumerate}

\begin{Shaded}
\begin{Highlighting}[]
\CommentTok{\# Double}
\end{Highlighting}
\end{Shaded}

\begin{enumerate}
\def\labelenumi{\arabic{enumi}.}
\item
  Create two new vectors called \texttt{my\_vec\_double} and
  \texttt{my\_vec\_int} which converts \texttt{my\_vec} to Double \&
  Integer types, respectively,
\item
  Create a new vector \texttt{my\_vec\_bool} which comprises of:

  \begin{itemize}
  \tightlist
  \item
    \texttt{TRUE}if an element in \texttt{my\_vec\_double} is \(\le 0\)
  \item
    \texttt{FALSE} if an element in \texttt{my\_vec\_double} is
    \(\ge 0\)
  \end{itemize}

  How many elements of \texttt{my\_vec\_double} are greater than zero?
\item
  Sort the values of \texttt{my\_vec\_double} in ascending order.
\end{enumerate}

\begin{center}\rule{0.5\linewidth}{0.5pt}\end{center}

\hypertarget{question-3}{%
\subsection{Question 3}\label{question-3}}

\begin{tcolorbox}[enhanced jigsaw, opacityback=0, rightrule=.15mm, leftrule=.75mm, colframe=quarto-callout-tip-color-frame, bottomrule=.15mm, opacitybacktitle=0.6, colbacktitle=quarto-callout-tip-color!10!white, toptitle=1mm, coltitle=black, bottomtitle=1mm, left=2mm, breakable, arc=.35mm, titlerule=0mm, title=\textcolor{quarto-callout-tip-color}{\faLightbulb}\hspace{0.5em}{50 points}, toprule=.15mm, colback=white]

\end{tcolorbox}

In this question we will get a better understanding of how \texttt{R}
handles large data structures in memory.

\begin{enumerate}
\def\labelenumi{\arabic{enumi}.}
\tightlist
\item
  Provide \texttt{R} code to construct the following matrices: \[
  \begin{bmatrix} 
  1 & 2 & 3\\
  4 & 5 & 6\\
  7 & 8 & 9\\
  \end{bmatrix}
  \quad \text{ and } \quad
  \begin{bmatrix} 
  1 & 2 & 3 & 4 & 5 & \dots & 100\\
  1 & 4 & 9 & 16 & 25 & \dots & 10000\\
  \end{bmatrix}
  \]
\end{enumerate}

\begin{tcolorbox}[enhanced jigsaw, opacityback=0, rightrule=.15mm, leftrule=.75mm, colframe=quarto-callout-warning-color-frame, bottomrule=.15mm, opacitybacktitle=0.6, colbacktitle=quarto-callout-warning-color!10!white, toptitle=1mm, coltitle=black, bottomtitle=1mm, left=2mm, breakable, arc=.35mm, titlerule=0mm, title=\textcolor{quarto-callout-warning-color}{\faExclamationTriangle}\hspace{0.5em}{Tip}, toprule=.15mm, colback=white]

Recall the discussion in class on how \texttt{R} fills in matrices

\end{tcolorbox}

In the next part, we will discover how knowledge of the way in which a
matrix is stored in memory can inform better code choices. To this end,
the following function takes an input \(n\) and creates an
\(n \times n\) matrix with random entries.

\begin{Shaded}
\begin{Highlighting}[]
\NormalTok{generate\_matrix }\OtherTok{\textless{}{-}} \ControlFlowTok{function}\NormalTok{(n)\{}
    \FunctionTok{return}\NormalTok{(}
        \FunctionTok{matrix}\NormalTok{(}
            \FunctionTok{rnorm}\NormalTok{(n}\SpecialCharTok{\^{}}\DecValTok{2}\NormalTok{),}
            \AttributeTok{nrow=}\NormalTok{n}
\NormalTok{        )}
\NormalTok{    )}
\NormalTok{\}}
\end{Highlighting}
\end{Shaded}

For example:

\begin{Shaded}
\begin{Highlighting}[]
\FunctionTok{generate\_matrix}\NormalTok{(}\DecValTok{4}\NormalTok{)}
\end{Highlighting}
\end{Shaded}

\begin{verbatim}
           [,1]       [,2]       [,3]        [,4]
[1,]  1.8589351  0.6678305 -0.3883626 -0.84096305
[2,] -1.5592177  1.4660626  0.9748198  0.02853166
[3,] -0.3274951 -2.1120681 -1.8218740  0.40217231
[4,] -1.3471276 -1.0477977 -0.5017016 -0.30035354
\end{verbatim}

Let \texttt{M} be a fixed \(50 \times 50\) matrix

\begin{Shaded}
\begin{Highlighting}[]
\NormalTok{M }\OtherTok{\textless{}{-}} \FunctionTok{generate\_matrix}\NormalTok{(}\DecValTok{50}\NormalTok{)}
\FunctionTok{mean}\NormalTok{(M)}
\end{Highlighting}
\end{Shaded}

\begin{verbatim}
[1] -0.01526334
\end{verbatim}

\begin{enumerate}
\def\labelenumi{\arabic{enumi}.}
\setcounter{enumi}{1}
\tightlist
\item
  Write a function \texttt{row\_wise\_scan} which scans the entries of
  \texttt{M} one row after another and outputs the number of elements
  whose value is \(\ge 0\). You can use the following \textbf{starter
  code}
\end{enumerate}

\begin{Shaded}
\begin{Highlighting}[]
\NormalTok{row\_wise\_scan }\OtherTok{\textless{}{-}} \ControlFlowTok{function}\NormalTok{(x)\{}
\NormalTok{    n }\OtherTok{\textless{}{-}} \FunctionTok{nrow}\NormalTok{(x)}
\NormalTok{    m }\OtherTok{\textless{}{-}} \FunctionTok{ncol}\NormalTok{(x)}

    \CommentTok{\# Insert your code here}
\NormalTok{    count }\OtherTok{\textless{}{-}} \DecValTok{0}
    \ControlFlowTok{for}\NormalTok{(...)\{}
        \ControlFlowTok{for}\NormalTok{(...)\{}
            \ControlFlowTok{if}\NormalTok{(...)\{}
\NormalTok{                count }\OtherTok{\textless{}{-}}\NormalTok{ count }\SpecialCharTok{+} \DecValTok{1} 
\NormalTok{            \}}
\NormalTok{        \}}
\NormalTok{    \}}

    \FunctionTok{return}\NormalTok{(count)}
\NormalTok{\}}
\end{Highlighting}
\end{Shaded}

\begin{enumerate}
\def\labelenumi{\arabic{enumi}.}
\setcounter{enumi}{2}
\tightlist
\item
  Similarly, write a function \texttt{col\_wise\_scan} which does
  exactly the same thing but scans the entries of \texttt{M} one column
  after another
\end{enumerate}

\begin{Shaded}
\begin{Highlighting}[]
\NormalTok{col\_wise\_scan }\OtherTok{\textless{}{-}} \ControlFlowTok{function}\NormalTok{(x)\{}
\NormalTok{    count }\OtherTok{\textless{}{-}} \DecValTok{0}
    
\NormalTok{    ... }\CommentTok{\# Insert your code here}

    \FunctionTok{return}\NormalTok{(count)}
\NormalTok{\}}
\end{Highlighting}
\end{Shaded}

You can check if your code is doing what it's supposed to using the
function here\footnote{If your code is right, the following code should
  evaluate to be \texttt{TRUE}

\begin{Shaded}
\begin{Highlighting}[]
\FunctionTok{sapply}\NormalTok{(}\DecValTok{1}\SpecialCharTok{:}\DecValTok{100}\NormalTok{, }\ControlFlowTok{function}\NormalTok{(i) \{}
\NormalTok{    x }\OtherTok{\textless{}{-}} \FunctionTok{generate\_matrix}\NormalTok{(}\DecValTok{100}\NormalTok{)}
    \FunctionTok{row\_wise\_scan}\NormalTok{(x) }\SpecialCharTok{==} \FunctionTok{col\_wise\_scan}\NormalTok{(x)}
\NormalTok{\}) }\SpecialCharTok{\%\textgreater{}\%}\NormalTok{ sum }\SpecialCharTok{==} \DecValTok{100}
\end{Highlighting}
\end{Shaded}
}

\begin{enumerate}
\def\labelenumi{\arabic{enumi}.}
\setcounter{enumi}{3}
\item
  Between \texttt{col\_wise\_scan} and \texttt{row\_wise\_scan}, which
  function do you expect to take shorter to run? Why?
\item
  Write a function \texttt{time\_scan} which takes in a method
  \texttt{f} and a matrix \texttt{M} and outputs the amount of time
  taken to run \texttt{f(M)}
\end{enumerate}

\begin{Shaded}
\begin{Highlighting}[]
\NormalTok{time\_scan }\OtherTok{\textless{}{-}} \ControlFlowTok{function}\NormalTok{(f, M)\{}
\NormalTok{    initial\_time }\OtherTok{\textless{}{-}}\NormalTok{ ... }\CommentTok{\# Write your code here}
    \FunctionTok{f}\NormalTok{(M)}
\NormalTok{    final\_time }\OtherTok{\textless{}{-}}\NormalTok{ ...  }\CommentTok{\# Write your code here}
    
\NormalTok{    total\_time\_taken }\OtherTok{\textless{}{-}}\NormalTok{ final\_time }\SpecialCharTok{{-}}\NormalTok{ initial\_time}
    \FunctionTok{return}\NormalTok{(total\_time\_taken)}
\NormalTok{\}}
\end{Highlighting}
\end{Shaded}

Provide your output to

\begin{Shaded}
\begin{Highlighting}[]
\FunctionTok{list}\NormalTok{(}
    \AttributeTok{row\_wise\_time =} \FunctionTok{time\_scan}\NormalTok{(row\_wise\_scan, M),}
    \AttributeTok{col\_wise\_time =} \FunctionTok{time\_scan}\NormalTok{(row\_wise\_scan, M)}
\NormalTok{)}
\end{Highlighting}
\end{Shaded}

Which took longer to run?

\begin{enumerate}
\def\labelenumi{\arabic{enumi}.}
\setcounter{enumi}{5}
\tightlist
\item
  Repeat this experiment now when:

  \begin{itemize}
  \tightlist
  \item
    \texttt{M} is a \(100 \times 100\) matrix
  \item
    \texttt{M} is a \(1000 \times 1000\) matrix
  \item
    \texttt{M} is a \(5000 \times 5000\) matrix
  \end{itemize}
\end{enumerate}

What can you conclude?

\begin{center}\rule{0.5\linewidth}{0.5pt}\end{center}

\hypertarget{appendix}{%
\section{Appendix}\label{appendix}}

Print your \texttt{R} session information using the following command

\begin{Shaded}
\begin{Highlighting}[]
\FunctionTok{sessionInfo}\NormalTok{()}
\end{Highlighting}
\end{Shaded}

\begin{verbatim}
R version 4.1.2 (2021-11-01)
Platform: x86_64-w64-mingw32/x64 (64-bit)
Running under: Windows 10 x64 (build 22000)

Matrix products: default

locale:
[1] LC_COLLATE=English_United States.1252 
[2] LC_CTYPE=English_United States.1252   
[3] LC_MONETARY=English_United States.1252
[4] LC_NUMERIC=C                          
[5] LC_TIME=English_United States.1252    

attached base packages:
[1] stats     graphics  grDevices datasets  utils     methods   base     

loaded via a namespace (and not attached):
 [1] compiler_4.1.2  fastmap_1.1.0   cli_3.6.0       htmltools_0.5.4
 [5] tools_4.1.2     yaml_2.3.7      rmarkdown_2.20  knitr_1.42     
 [9] xfun_0.36       digest_0.6.31   jsonlite_1.8.4  rlang_1.0.6    
[13] renv_0.16.0-53  evaluate_0.20  
\end{verbatim}



\end{document}
